\section{Summary} \label{sec:abstract}

The \gls{LSST} is expected to discover millions of Solar System objects (SSOs) as reported in Table \ref{tab:table1}.
Roughly 2bn observations of SSOs can be expected after ten years of the \gls{LSST} survey with up to 100,000 new discoveries per night. Most objects are going to be discovered during the first years of \gls{LSST} operations. Main Belt Asteroids (MBAs) are the largest contributors to both, \gls{SSO} discoveries and operations.
The corresponding nightly data volumes to be delivered to the Minor Planet \gls{Center} are on the order of hundreds of MBs and only rarely exceed a \gls{GB}.
Even with near perfect linking and detection algorithms driving the Solar System Processing (\gls{SSP}) pipelines, not all \gls{SSO} observations can be attributed to known sources or new discoveries.
The fraction of \gls{LSST} \gls{SSO} observations that fall into that category decreases over time, but is unlikely to fall below 2.5\% (50M) even in the best case scenario of finding every discoverable asteroid at the first opportunity. Roughly half of those observations belong to objects that are not discoverable at all given the current discovery strategy of ``three nighters", i.e. two pairs of observations over three nights in a 15-30 day window. Those will be reported as unattributed transients in a roughly 50deg wide \gls{declination} band around the ecliptic. A delayed start in processing Solar System observations, caused by a lack of templates in the first year, for instance, would not significantly affect Main Belt Asteroid discovery as long as data is collected and computational resources similar to the yearly data reprocessing can be made available at a later time. Timely follow-up observations of near-Earth asteroids and interstellar objects would become impossible during the delayed processing period, however.

\begin{table}[tb!]
\begin{center}
\caption{Summary of small body populations observed with \gls{LSST}.}
\label{tab:table1}
 {\scriptsize
  \begin{tabular}{|l|c|c|}\hline\hline
Population & known as of 10/2019 &  \gls{LSST} discoveries$^{(1)}$ \\ \hline
Near-Earth Objects (NEOs) & 21,172 & 49,000-93,000  \\\hline
Main Belt Asteroids (MBAs) & 796,354 & 5,400,000-225,000,000$^{(*)}$  \\ \hline
Jupiter Trojans & 7,384 & 280,000$^{(+)}$\\ \hline
TransNeptunian and Scattered Disk Objects (TNOs and SDOs) & 3,800 & 40,000$^{(+)}$ \\ \hline\hline
 \end{tabular}
  }
 \end{center}
\vspace{1mm}
 \scriptsize{
 {\it Notes:}\\
  $^{(1)}$ : Expected at the end of \gls{LSST}'s ten years of operations.
  $^{(*)}$ : the upper bound is based on an (unlikely) \gls{TNO} like size-frequency distribution.
  100\% completeness (as indicated).
  $^{(+)}$ : Not updated from \citep{jones2015asteroid}.}
\end{table}


\section{Introduction} \label{sec:intro}
\begin{figure}[tb!]
\begin{center}
\includegraphics[scale=0.27]{./figs/mops2.png}
\end{center}
\caption{Concept of the \gls{LSST} Solar System Processing cycle.}
\label{fig:mops}       % Give a unique label
\end{figure}
%During operations the LSST will produce roughly 20TB of data per night.
The \gls{LSST} is expected to discover millions of Solar System objects (SSOs) over the course of its ten years of operation \citep{jones2015asteroid}.
Past assessments of discovery yields of future surveys see the \gls{LSST} becoming the dominant contributor of Solar System \gls{Object} (\gls{SSO}) related data to the \gls{IAU} Minor Planet \gls{Center} (\gls{MPC}) \citep{LSSTscibook2009}. The currently agreed upon strategy sees \gls{LSST} deliver astrometric and photometric observations of
known SSOs as well as observations that represent potential discoveries after every night of operations.
Figure \ref{fig:mops} shows a corresponding schematic of the proposed \gls{LSST} \gls{SSO} processing \gls{pipeline}.
Apart from Space Situational Awareness (\gls{SSA}) and planetary defense agencies that are mostly concerned with the capabilities of the \gls{LSST} to discover near-Earth asteroids, services that ingest and process \gls{LSST} data on a regular basis, such as the \gls{MPC},
have a great interest in estimates regarding the predicted data \gls{flux} directed their way. This requires accurate estimates regarding the actual number of predicted \gls{SSO} observations and discoveries. Early estimates stem from the \gls{LSST} Science Book \citep{LSSTscibook2009}.
%\begin{table}
%\begin{center}
%\caption{Summary of small body populations observed with LSST as of 2015, from \citep{jones2015asteroid}.}
%\label{tab:table1}
% {\scriptsize
%  \begin{tabular}{|l|c|c|}
%  \hline\hline
%Population & known as of 2015 &  LSST discoveries$^1$ \\ \hline
%Near Earth Objects (NEOs) & 12,832 & 100,000   \\\hline
%Main Belt Asteroids (MBAs) & 636,499 & 5,500,000 \\ \hline
%Jupiter Trojans & 6,387 & 280,000  \\ \hline
%TransNeptunian and Scattered Disk Objects (TNOs and SDOs) & 1,921 & 40,000  \\ \hline\hline
% \end{tabular}
%  }
% \end{center}
%\vspace{1mm}
% \scriptsize{
% {\it Notes:}\\
%  $^1$)Expected at the end of LSST's ten years of operations.}
%\end{table}
Some progress has been made in this respect, both in scientific literature and in the \gls{LSST} project. Hence, it is the aim of this \gls{DMTN} to summarize and, where necessary, to provide updated \gls{LSST} \gls{SSO} observation and discovery rates as well as their uncertainties. This information is intended to inform internal and external services affected by \gls{SSO} discoveries.
The remainder of this \gls{DMTN} is structured as follows: Section \ref{sec:lsstss} briefly describes the \gls{LSST} survey and the simulation setup. Section \ref{sec:neo} provides a summary of previous work as well as a brief update on near-Earth asteroid discovery with the \gls{LSST}. Section \ref{sec:mba} discusses \gls{LSST} capabilities to discover Main Belt Asteroids that constitute the main source of \gls{LSST} \gls{SSO} observations. In section \ref{sec:delay} the impact of delayed \gls{LSST} operations and Solar System processing on discovery requirements is studied. Section \ref{sec:noise} assesses how much noise unattributed \gls{SSO} observations could contribute to \gls{LSST} data. As \gls{MBA} observations dominate \gls{SSO} observations section \ref{sec:data} discusses likely data rates that \gls{LSST} has to submit to the \gls{IAU} Minor Planet \gls{Center} (\gls{MPC}) on a nightly basis. Section \ref{sec:other}  looks at other \gls{SSO} populations and section \ref{sec:conclusions} concludes this report.

%According to the MPC, ingesting additional information on known objects can be performed with relatively little effort.
%Confirming and providing preliminary orbits for potential discoveries, on the other hand, requires more resources.
%
%Updated predictions on how many SSOs LSST will discover
%are, thus, of great interest, since they can serve as guideline for future infrastructure requirements.
\section{LSST survey simulation} \label{sec:lsstss}
The proposed observation strategy for the \gls{LSST} survey used in this work is described in \href{http://astro-lsst-01.astro.washington.edu:8080/multiColor?runId=1#Fivesigmadepth}{baseline2018}. Its sky coverage, inter and intra-night gaps as well as the performance in terms of median and maximum five sigma depth is shown in Figure \ref{fig:gaps}.

\begin{figure}[tb!]
\begin{tabular}{c}
\includegraphics[width=1\linewidth]{figs/baseline2018a.png}
\end{tabular}
\caption{LSST survey simulation "baseline2018a". The left column shows intra- (top) and inter-night gaps (bottom) between \gls{LSST} returns to the same pointing. In the right column median (top) and maximum (bottom) five sigma depths are shown.}
\label{fig:gaps}       % Give a unique label
\end{figure}
%
During the \gls{Wide-Fast-Deep} (WDF) survey \gls{LSST} is expected to reach median five sigma depths per image around \SI{23.5}{mag} with median inter and intra-night gaps of 1.5 days and 20 mins, respectively. Such a cadence facilitates the construction of nightly tracklets, successive observations of
moving objects spanning a short arc. Tracklets of individual objects can then be linked over several nights resulting in the discovery of SSOs.
Two types of survey simulation software frameworks were used to generate and validate \gls{LSST} \gls{SSO} discovery statistics, namely
\begin{itemize}
\item \href{https://epyc.astro.washington.edu/~lynnej/sims_movingObjects/lsst.sims.movingObjects.html}{LSST Sims\_movingObjects (SIMS\_MO)}
\item \href{https://github.com/eggls6/openobs}{Open Observation Simulator (OPENOBS)}
\end{itemize}
The former is a high fidelity simulator that includes N-body numerical orbit propagation, the actual \gls{LSST} \gls{camera} model, as well as color information of asteroids. SIMS\_MO is capable of performing accurate observation simulations for up to several 100k of SSOs. Although that constitutes only a fraction of the true \gls{MBA} population, information on expected survey completeness statistics can be extracted through sampling of individual orbits over a grid of absolute magnitudes.
OPENOBS is capable of performing survey simulations on the entire expected \gls{MBA} population (around 100M objects) albeit with a lower fidelity. Instead of N-body propagation unperturbed heliocentric Keplerian orbits of SSOs are assumed. The \gls{LSST} \gls{camera} \gls{footprint} is simulated, but the exact geometry including chip gaps has not been taken into account in OPENOBS simulations. Instead, the fill factor of the \gls{camera} system as calculated in \citet{veres2017high} was used to statistically filter observations. Spectral energy distributions (SEDs) of asteroids are assumed to be flat in OPENOBS simulations, while S- and C-type asteroid \gls{SED}s are used in SIMS\_MO (Figure \ref{fig:sed}). The magnitude cutoff in our simulation was chosen to be H=25mag corresponding to asteroid diameters between 19m-60m depending on object albedo (0.5-0.05).
\begin{figure}[tb!]
\begin{center}
\includegraphics[width=0.5\textwidth]{figs/5sig_filter.png}\\
\includegraphics[width=0.5\textwidth]{figs/5sig_filter_S.png}\\
\includegraphics[width=0.5\textwidth]{figs/5sig_filter_C.png}
\end{center}
\caption{A comparison of \gls{LSST} depths per filter for a flat \gls{SED} (top), an \gls{SED} corrected for S-type asteroids (center) and C-type asteroids (bottom), respectively. The number of \gls{LSST} visits is shown on the y-axis. Values on the x-axis are given in magnitudes.}
\label{fig:sed}       % Give a unique label
\end{figure}
%Combining results from SSO population studies and LSST's expected performance we suggest that the
%MBAs will dominate SSO discoveries.
%
%800k Mag < H
%
% trailing losses
% probabilistic 50\%
% default uses fading
% vignetting
% camera orientation is includeded
\clearpage
\section{Near-Earth Asteroids}\label{sec:neo}
Substantial efforts have been undertaken to assess the potential of the \gls{LSST} for discovering near-Earth asteroids \citep[e.g.][]{ivezic2006lsst,jones2015asteroid,grav2016modeling,veres2017near,veres2017high,jones2018large,ivezic2019lsst}. The current consensus is that \gls{LSST} is capable of achieving between 60 and 65\% integral completeness NEOs with absolute magnitude brighter than 22, roughly 140m in size and larger. Together with other existing \gls{NEO} surveys that number can be increased to 73\% with the current \gls{LSST} survey strategy. The prospects for discovering Potentially Hazardous Asteroids (PHAs) are even better, with a predicted 81\% completeness at the end of the nominal \gls{LSST} survey \citep{ivezic2019lsst}. The completeness fraction is largely independent of \gls{NEO} population models such as \citep[S3M,][]{s3m} and \citet{granvik2018neos}, see Figure \ref{fig:neo_compl}. In contrast, the actual number of discovered objects varies considerably as shown in Figure \ref{fig:neo_count}. \gls{LSST} discoveries for asteroids brighter than H=25mag range from 50k to 100k over the 10yrs of the \gls{LSST} survey, which corresponds to 2.5 to 5 times the number of currently known NEOs (Figure \ref{fig:neo_current}).
Uncertainties in the \gls{NEO} population model are the most significant contributors to the uncertainty in the number of \gls{NEO} discoveries. All \gls{NEO} simulations have been performed with SIMS\_MO.
\begin{figure}[tb!]
\begin{center}
\includegraphics[width=0.65\linewidth]{figs/neo2.png}
\end{center}
\caption{Cumulative near-Earth asteroid (NEA) and Potentially Hazardous Asteroid (\gls{PHA}) completeness at the end of the 10yr \gls{LSST} survey for different asteroid population models \citep[S3M,][]{s3m} and \citet{granvik2018neos}. Results for three pairs of observations over 15 nights are compared to similar detection conditions over 30 nights.}
\label{fig:neo_compl}       % Give a unique label
\end{figure}

\begin{figure}[tb!]
\begin{center}
\includegraphics[width=0.65\linewidth]{figs/neo1.png}
\end{center}
\caption{Number of NEOs discovered with the \gls{LSST} for different population models \citep[S3M,][]{s3m} and \citet{granvik2018neos}.
The prediction curves represent the cumulative number of discovered NEOs as a function of their absolute magnitude H, while the model curves show the number of expected objects in the population.}
\label{fig:neo_count}       % Give a unique label
\end{figure}

\begin{figure}[tb!]
\begin{center}
\includegraphics[width=0.65\linewidth]{figs/nea_vs_time_chart.png}
\end{center}
\caption{Number of known NEOs as a function of time and approximate size. The ``All" category contains objects of all sizes.\gls{Source}: \gls{NASA} \gls{JPL}/Caltech}
\label{fig:neo_current}       % Give a unique label
\end{figure}

\clearpage

\section{Main Belt Asteroids} \label{sec:mba}
The vast majority of SSOs accessible to the \gls{LSST} populate the Main-Belt (\gls{MB}) between Mars and Jupiter \citep{LSSTscibook2009}. Estimates on \gls{SSO} discovery and observation statistics presented in this document are, therefore, based on simulations of Main-Belt asteroid (\gls{MBA}) observations with the \gls{LSST}.
In order to quantify the number of potential \gls{LSST} discoveries, we adopt a simple model for the size-frequency and orbit distribution of MBAs of the form
\begin{equation}
 N(\text{H})=\beta\;10^{\;\alpha \text{H}}
 \label{eq:lsw}
\end{equation}
where $N$ is the number of MBAs, H the absolute magnitude, and $\alpha$ and $\beta$ determine the slope and the anchor value of the log-linear fit, respectively.
The parameters in equation (\ref{eq:lsw}) were derived from the current \gls{MBA} population using actual data of the main belt supplied by the \gls{MPC}.
Assuming the bright main belt population is essentially complete up to H$=$14.5mag, we can fit lower and upper bounds on the size-frequency distribution that we extrapolate to fainter objects. The corresponding results are presented in Figure \ref{fig:mba_pop} as well as in Table \ref{tab:mbapop}.
In order to guarantee that the slope and anchor parameters for the \gls{MBA} population are not influenced by the specific choice of the bin-size in H we derived the underlying distribution $n$(H) using kernel density estimation where
\begin{equation}
 n\;(\text{H})=\frac{1}{I\,b} \sum_{i=1}^{I} K\left(\frac{\text{H}-\text{H}_i}{b}\right),
 \label{eggl:eq:kernel}
\end{equation}
where $K$ is the (Gaussian) kernel and $b$ a smoothing parameter, respectively. The resulting slopes are compatible with \citet{jedicke2002} and \citet{heinze2019flux} who find local slopes between $0.24<\alpha<0.58$ for MBAs with H$>10$mag, as well as with size-frequency distributions from Trans-Neptunian Objects (TNOs), where $\alpha\approx0.66$ \citep{bernstein2004}.
Orbital elements of synthetic MBAs are sampled from the joint distribution of orbital elements of the current main belt. We assume that the size-frequency and orbital element distribution are essentially uncorrelated. While, strictly speaking, this is not the case for asteroids that belong to families, a similar approach was used to create the S3M \citep{s3m}, a synthetic Solar System model for \gls{Pan-STARRS}.
In contrast to the "bracketing" approach adopted in this work via estimating upper and lower bounds to the size-frequency distribution in the main belt, the \gls{MBA} population in the S3M is based on work by \citet{jedicke2002}. A comparison between the here adopted model and \citet{jedicke2002} used in S3M can be seen in Figure \ref{fig:mba_pop}.
\begin{table}[t!]
\begin{center}
\begin{tabular}{cllr}
\hline
model & $\alpha$ & $\beta$ & $N_{tot}$(H $\le$ 25mag) \\
\hline
A & 0.3 &  2.29 & 104 $\times 10^{6}$ \\
B & 0.56 &4.43 $\times 10^{-4}$ & 34.5$\times 10^{9}$\\
\hline
\end{tabular}
\end{center}
\caption{Main Belt Asteroid (\gls{MBA}) population models used in this work. The parameters $\alpha$ and $\beta$ are input to equation (\ref{eq:lsw}). $N_{tot}$ is the estimate for the total asteroid population with $H\le 25$mag. }
\label{tab:mbapop}
\end{table}
%%%
\begin{figure}[b!]
%\includegraphics[scale=.65]{pics/ptype_gf_a1.png}
\begin{center}
\includegraphics[scale=.8]{figs/mpc_population_4.png}
\end{center}
\caption{MBA population extrapolation using a continuous (red, green) and broken H($N$) power law (orange), respectively.}
\label{fig:mba_pop}       % Give a unique label
\end{figure}
%

The current strategy of the \gls{LSST} Solar System Processing is to send only those observations of SSOs to the \gls{MPC} that have a high likelihood of being either associated to know objects or new discoveries.
In order to minimize false positives in potential asteroid discoveries, observations of new objects are only submitted, if they are detected at least twice per night on three distinct nights (so-called "three nighters"). The three nights do not have to be consecutive. Detections can be spread over a window of 15-30 nights. Assuming a linking efficiency close to 100\%, such "three-nighters" can be turned into preliminary orbits that have a high likelihood of belonging to newly discovered objects.
The expected completeness as a function of absolute magnitude (a proxy for the size of objects) for MBAs is shown in Figure \ref{fig:mba_compl} after 6 months, 1 year, 2 years and after 10 years of \gls{LSST} operations.
\begin{figure}[b!]
%\includegraphics[scale=.65]{pics/ptype_gf_a1.png}
\begin{center}
\includegraphics[scale=0.7]{figs/mba_completeness3.png}
\end{center}
\caption{Survey completeness for MBAs at a given absolute magnitude (H) as a function of time.}
\label{fig:mba_compl}       % Give a unique label
\end{figure}
%
The graph shows that after two years of survey more than 90\% of kilometer sized MBAs will have been discovered.
By the nominal end of the \gls{LSST} survey we know essentially all MBAs with diameters of half a kilometer and larger.
The total number of discovered MBAs as a function of absolute magnitude is presented in Figure \ref{fig:mba_disc}. Results for the two  size-frequency distributions probed are
shown for both software packages, SIMS\_MO and OPENOBS. Formal errors for SIMS\_MO discoveries are based on the sample size selected from the assumed size frequency distribution.
A comparison of Figures \ref{fig:mba_compl} and \ref{fig:mba_disc} shows that results for SIMS\_MO and OPENOBS are very similar where \gls{LSST} is expected to be essentially complete (H$\approx19$mag). For smaller and fainter objects discovery rates differ slightly leading to a discrepancy of roughly 10\%, which is negligible compared to the uncertainties in the expected number of MBAs.
%
\begin{figure}[tb!]
\begin{center}
\includegraphics[scale=0.9]{figs/mba_disc3.png}
\end{center}
\caption{LSST Main Belt asteroid discoveries as a function of absolute magnitude for two different size-frequency distributions (population model A and \gls{B}) and simulation software packages OPENOBS and SIMS\_MO. See text for details.}
\label{fig:mba_disc}       % Give a unique label
\end{figure}
%
Figure \ref{fig:mba_disc_stats} shows the expected number of \gls{LSST} \gls{MBA} discoveries per night as predicted by OPENOBS simulations. The number of discovered objects ranges from a few per night to 100,000 discoveries depending on the momentary cadence and ecliptic latitude. Figure \ref{fig:mba_disc_stats} provides statistical information on how many nights are expected to yield a certain number of discoveries. The conservative size-frequency distribution sees an average of 2000 discoveries per night\footnote{not observed night} during the ten year survey for more than 200 nights. Nights with more objects are rare but do occur. Discovery maxima of roughly 100,000 discoveries per night can occur. A steeper size-frequency distribution of MBAs (population model \gls{B}) would see nights with more than 10 million discoveries.

Figure \ref{fig:mba_disc_yr} shows the number of expected \gls{MBA} discoveries for each of the ten years of \gls{LSST} operations. If Solar System Processing (\gls{SSP}) starts right away, the majority of objects will be discovered during the first years of the \gls{LSST} survey. Figures \ref{fig:mba_compl} and \ref{fig:mba_disc_yr} demonstrate the power of the \gls{LSST} to quickly distinguish between various hypothesis on the size-frequency distribution in the main belt.
%
\begin{figure}[tb!]
\begin{center}
\begin{tabular}{cc}
\includegraphics[width=0.5\linewidth]{figs/disc_per_night.png} &
\includegraphics[width=0.50\linewidth]{figs/disc_stat.png}
\end{tabular}
\end{center}
\caption{Left: \gls{LSST} Main Belt Asteroid (\gls{MBA}) discoveries per observed night for population model A. Right: Number of nights with a certain discovery yield versus number of discoveries per night for two different \gls{MBA} size-frequency distributions.}
\label{fig:mba_disc_stats}       % Give a unique label
\end{figure}

\begin{figure}[tb!]
\begin{center}
\begin{tabular}{cc}
\includegraphics[width=0.5\linewidth]{figs/discovery_time.png} &
\includegraphics[width=0.5\linewidth]{figs/disc_per_yr2.png}
\end{tabular}
\end{center}
\caption{LSST Main Belt Asteroid (\gls{MBA}) discoveries per year on a linear (left) and logarithmic scale (right) assuming 100\% linking efficiency. The left panel compares yields for early and late discovery scenarios. The right panel shows discovery yields for population models A (green) and \gls{B} (red) in an early discovery scenario.}
\label{fig:mba_disc_yr}       % Give a unique label
\end{figure}
%
The on-sky distribution of observations of discoverable MBAs is shown in Figure \ref{fig:mba_obs_sky}. The darker spots showing an increased number of observations per square degree correspond to the \gls{LSST} deep drilling fields from small to high Right Ascension values: ELAIS S1, \gls{XMM}-LSS, Extended Chandra, Deep Field-South and COSMOS.
\begin{figure}
\begin{center}
\includegraphics[width=0.70\linewidth]{figs/mba_obs_hpmap.png}
\end{center}
\caption{On-sky (ICRF) distribution of observations per square degree of MBAs (population model A, 14.5 $\le$ H [mag] $\le$ 25) discoverable with the \gls{LSST}. The dark spots suggesting a locally larger number of observations correspond to the \gls{LSST} deep drilling fields from right to left: ELAIS S1, \gls{XMM}-LSS, Extended Chandra, Deep Field-South and COSMOS.}
\label{fig:mba_obs_sky}       % Give a unique label
\end{figure}
%
Many asteroids are observed frequently enough so that they can be discovered later on, if initial discovery opportunities were missed (Figure \ref{fig:do}). Missed discovery opportunities may result from a failure to identify and/or link possible tracklets in the Solar System Processing \gls{pipeline}, or a hardware/software malfunction in the telescope itself, for instance. Bright objects have on average 200 discovery opportunities. This number drops fast for fainter objects, however. Discovery opportunities for MBAs generally reoccur after a synodic period between the Earth and the Main belt ($\approx 500$days) as can be seen from Figure \ref{fig:do}. Most of the very faint objects have one discovery opportunity only. They populate the main diagonal in the right panel of Figure \ref{fig:do}.
%
\begin{figure}[tb!]
\begin{tabular}{ll}
\includegraphics[width=0.5\linewidth]{figs/disc_opport3.png}
\includegraphics[width=0.5\linewidth]{figs/first_vs_last.png} &
\end{tabular}
\caption{Left: Discovery opportunities for MBAs as a function of their absolute magnitude (H). The red line indicates the median for each H bin. Right: First vs. last discovery opportunity for MBAs. The color coding indicates the \gls{LSST} night in which the object could be first discovered.}
\label{fig:do}       % Give a unique label
\end{figure}
\clearpage
%%%%%%%%%%%%%%%%%%%
\section{Impact of delayed \gls{LSST} operations and/or Solar System Processing} \label{sec:delay}
%
\begin{figure}[tb!]
\begin{center}
\includegraphics[width=0.70\linewidth]{figs/reprocessing4.png}
\caption{Cumulative number of MBAs (blue) and observations (red) that require reprocessing if \gls{SSP} is delayed. }
\end{center}
\label{fig:rep}       % Give a unique label
\end{figure}
Assuming \gls{LSST} operations proceed nominally and data of \gls{SSO}s is collected a delay in \gls{SSP} would mostly impact \gls{NEO}, interstellar objects (ISOs) and Solar System \gls{transient} science. This is due to the fact that the success of follow up observations that are curcial to coroborate discoveries hinges on a timely identification and publication of objects of interest.
Discovery and attribution of \gls{LSST} observations of other groups is less time sensitive of \gls{SSO}s could be performed at a later date using archived data.
As can be gathered from Figure \ref{fig:rep}, an \gls{SSP} delay of a month requires the reprocessing of almost half a million objects with 20 million attributable observations. If \gls{SSP} were to be delayed for a year, for instance, to make high quality templates available for image differencing, around 100 million observations of two million objects would need to be processed.
While the computational resources to recover from delayed \gls{SSP} are significant, similar work is already expected as part of the yearly data releases.
Hence, we do not anticipate a change in the computational requirements at this point, even if the \gls{SSP} were to be delayed.
%
\begin{figure}[tb!]
\begin{center}
\includegraphics[width=0.70\linewidth]{figs/lost4.png}
\end{center}
\caption{Losses in \gls{MBA} discoveries due to delayed operations of the \gls{LSST} (assuming no extension) for population model A.}
\label{fig:lost}       % Give a unique label
\end{figure}
In a worst case scenario where the entire \gls{LSST} operations are delayed, the majority of asteroids could still be discovered during the last years of the nominal period of operations (Figures \ref{fig:mba_disc_yr} and \ref{fig:lost}).\footnote{Note, that this is not necessarily true for near-Earth asteroids, since their discovery function is not as simple a function of the synodic period as it is for MBAs.}
A delay in \gls{LSST} operations for a month would merely result in 0.1\% of asteroids lost. Shortening operations by a
a year would result in the loss of 2\% of main belt asteroid discoveries, roughly 200k objects.

\clearpage

\section{Solar System Objects as a source of alerts} \label{sec:noise}
\begin{figure}[tb!]
\begin{center}
\begin{tabular}{cc}
\includegraphics[width=0.5\linewidth]{figs/cum_un_obs_first.png} &
%\includegraphics[width=0.45\linewidth]{figs/cum_un_obs_last.png}\\
\includegraphics[width=0.5\linewidth]{figs/cum_un_obs_first2.png}\\
%\includegraphics[width=0.45\linewidth]{figs/cum_un_obs_last2.png}\\
\end{tabular}
\end{center}
\caption{Cumulative number of \gls{LSST} observations of MBAs over time. The left panels shows all \gls{LSST} observations (blue), observations of bright, already known objects (orange), unassociated observations in tracklets (cyan) and singlets (green) and non-discoverable objects in trackelts (red) and singlets (purple).
%The right panels display the corresponding curves for the case where all asteroids are discovered at the last instead of the first opportunity.
The left panel is log linear, while the right panel contains a log-log plot. The corresponding numbers are presented in Table \ref{tab:mbaobs}.\label{fig:cum_un_obs}}
       % Give a unique label
\end{figure}
%
Not all new SSOs detected by the \gls{LSST} accumulate enough observations to count as ``discovered".
The current discovery policy for SSOs requires repeated recurrence (three-nighters).
An object must have at least two observations per night over three nights in a 15-30 night window to be considered a candidate object that can be sent to and validated by the \gls{MPC}.
Detections that can neither be linked to known objects nor grouped to form a new discovery will still trigger alerts.
In order to estimate how many observations of SSOs belong to that latter category we used the baseline2018b cadence together with the most likely size-frequency distribution for
asteroids in the main belt (model A) and simulated observations via OPENOBS. The resulting observations, roughly 2 billion, were then analyzed in order to understand how many observations could be linked to known and newly discovered sources and how many would remain unassociated.
The outcome is presented in Figures \ref{fig:cum_un_obs} as well as in Table \ref{tab:mbaobs}.
\begin{figure}[tb!]
\begin{center}
\begin{tabular}{c}
\includegraphics[width=0.50\linewidth]{figs/unasoc_frac_first_chance3.png}
\includegraphics[width=0.50\linewidth]{figs/unasoc_frac_last_chance3.png}
\end{tabular}
\end{center}
%\caption{Fraction of unassociated and non-discoverable object observations of MBAs assuming all asteroids have been discovered at the first opportunity.  \label{fig:ufc} }
\caption{The fraction of unassociated and non-discoverable object observations with respect to all \gls{MBA} observations is shown. In the graph on the left we assume all asteroids have been discovered at the first opportunity. In the graph on the right all asteroids have been discovered at the last opportunity. \label{fig:ufc} }
\end{figure}
We distinguish between observations of
\begin{itemize}
\item bright objects that are known today will account for approximately 30\% of \gls{LSST} \gls{SSO} observations,
\item unassociated observations, i.e. observations objects that have not yet been discovered at the \gls{epoch} of observation,
\item and observations of non-discoverable objects (NDOs) that are too sparse to be grouped into "three nighters".

\end{itemize}
%
\begin{figure}[tb!]
\begin{center}
\includegraphics[width=0.7\linewidth]{figs/nights_vs_obs.png}
\end{center}
\caption{Frequency of nights with a given number of observations assuming all asteroids are discovered at the first opportunity.\label{fig:obs_stats}}
       % Give a unique label
\end{figure}
Figure \ref{fig:ufc} shows that the majority (97.4\%) of all \gls{LSST} observations of asteroids can be attributed to SSOs after ten years of operations assuming all asteroids are discovered with the first opportunity. The data displayed in Table \ref{tab:mbaobs} underlines this fact.
However, the fraction of unassociated observations remains non-negligible throughout the survey.
Figure \ref{fig:ufc} shows that that fraction is around 15\% a year into the \gls{LSST} survey compared to 2.5\% near the end. This is not surprising as the fraction of discovered to undiscovered objects increases over time.
If asteroids are not discovered at the first opportunity, the fraction of unassociated observations increases. 
Even if all asteroids are discovered in the end, missed discovery opportunities increase the number of unassociated observations substantially, as can be seen in the right panel of Figure \ref{fig:ufc}. That scenario assumed all asteroids would be discovered at the last discovery opportunity.


Precovery algorithms that are applied to already collected data can reduce the number of unassociated observations by roughly 50\%, but only a posteriori.
At the time of observation, unassociated observations may still trigger \gls{LSST} alerts.
For the likely case of early asteroid discovery Figure \ref{fig:obs_stats} suggests that most nights tend to have a relatively low number of unassociated observations, typically between 1k-10k (Table \ref{tab:obs_stats}). Similarly, one can expect between 100 and 10k observations from non-discoverable objects per night. Only a hand-full of nights may prove to be ``noisier".
Figure \ref{fig:ndo_h} shows the absolute magnitude distribution of detected, discovered and non-discovered objects for the \gls{MBA} population model A.
One can see that the \gls{LSST} \gls{MBA} survey is likely to discover practically all objects down to absolute magnitude H$=19.5$mag.
Absolute magnitudes for observed, non-discoverable objects range from H=19mag to H=25mag.
The on-sky distribution of observations of non-discoverable objects is shown in Figure \ref{fig:ndo_obs_sky}. Over ten years of operations, one can expect fewer than 4000 observations per square degree from non-discoverable MBAs near the ecliptic with an all-sky median value of 76 per night (\ref{tab:obs_stats}). The number of unassociated observations would be about three times higher, roughly 12,000 per square degree by the end of the survey.
%
\begin{table}[tb!]
\begin{center}
\begin{tabular}{lccccc}
\multicolumn{5}{c}{LSST Observations of MBAs}\\
\hline\hline
per night & all obs. & discoveries & known obj. & \gls{UNO} & \gls{NDO} \\\hline
median  & 528,022 &312,077  &177,172 &626 &72 \\
mean & 621,084 & 402,334 & 193,570 & 17,490 & 7,688 \\
std. dev. & 511,835 & 363,013& 145,676 &47,643 & 24,491\\
maximum & 2,976,668 & 2,134,030 & 683,392 & 656,687 & 356,857\\
\hline\hline\\
\end{tabular}
\end{center}
\caption{Nightly statistics for all \gls{LSST} \gls{MBA} observations, currently known object observations, unassociated observations (\gls{UNO}) and observations of
non-discoverable objects (\gls{NDO}).}
\label{tab:obs_stats}
\end{table}
%

\begin{figure}[tb!]
\begin{center}
\begin{tabular}{cc}
\includegraphics[width=0.70\linewidth]{figs/detection_discovered3.png}
\end{tabular}
\end{center}
\caption{Number of detected, discovered and non-discovered MBAs as a function of their absolute magnitude H. }
\label{fig:ndo_h}       % Give a unique label
\end{figure}

\begin{table}[tb!]
\begin{center}
\begin{tabular}{lcccccc}
\multicolumn{7}{c}{LSST Observations of MBAs}\\
\hline
Discovery at & Total & $H<14.5$mag & \gls{UNO} TR & \gls{UNO} SNGL & \gls{NDO} TR & \gls{NDO} SNGL \\
\hline
first chance & 2.27e+09 & 7.06e+08 & 5.39e+07 & 9.86e+06 & 2.37e+07 & 4.36e+06 \\
percentage   & 100\%  & 31.2\% & 2.4\% & 0.4\% & 1.1\% & 0.2\% \\\hline
%last chance & - & - & 1.18e+09 & 1.17e+08 & - & -  \\
%percentage   & - & - & 52.1\% & 5.2\% & -  & -\\
\hline
\end{tabular}
\end{center}
\caption{Main Belt Asteroid (\gls{MBA}) observations at the end of the \gls{LSST} survey. Observations of bright, already known objects are in the column $H<14.5$mag.
Unassociated observations (\gls{UNO}) can partly be linked to discovered objects via \gls{precovery}. NDOs are observations of non-discoverable objects. The number of observations that are part of tracklets (TR) is contrasted with single observations per night (SNGL).\label{tab:mbaobs}}
\end{table}

\begin{figure}[tb!]
\begin{center}
\begin{tabular}{cc}
\includegraphics[width=0.80\linewidth]{figs/mba_obs_hp_ndo.png} &
\end{tabular}
\end{center}
\caption{On-sky distribution of observations per square degree of non-discoverable objects (NDOs).  }
\label{fig:ndo_obs_sky}       % Give a unique label
\end{figure}


\section{Expected \gls{MPC} Data Rates}\label{sec:data}
Based on observation statistics presented in the previous section we can estimate nightly data rates for submitting \gls{SSO} observations to the \gls{MPC}. The new Astrometry Data Exchange Standard (\gls{ADES}) submission format\footnote{\url{https://www.minorplanetcenter.net/iau/info/ADES.html}} that will be adopted by the time \gls{LSST} goes on line permits a more detailed description of optical observations of SSOs. Observation covariance information can be provided, for instance. As such, each observation should have an uncompressed size of roughly 500bytes when submitted to the \gls{MPC}. Figure \ref{fig:data} shows the expected nightly data volumes. Submissions will typically be on the order of several hundred MBs per night. \gls{GB} sized submissions are fairly rare occurring on roughly 50 nights during the ten years of \gls{LSST} operations. Given the fairly steep size-frequency distribution in population model A that has been used in this assessment, observations of new discoveries will dominate \gls{MPC} submissions despite the fact that known objects are brighter and tend to have more observations per object. Table \ref{tab:data_stats} contains the corresponding statistics. It is currently not planned to submit all observations to the \gls{MPC}. Only observations
that can be linked to known objects and those that can be confidently attributed to new discoveries are to be delivered.


\begin{figure}[tb!]
\begin{center}
\includegraphics[width=0.70\linewidth]{figs/data2.png}
\end{center}
\caption{Predicted nightly data volumes for submissions of \gls{LSST} observations to the Minor Planet \gls{Center}.}
\label{fig:data}       % Give a unique label
\end{figure}
\clearpage

\begin{table}[tb!]
\begin{center}
\begin{tabular}{lccc}
\multicolumn{4}{c}{LSST data volumes for \gls{MPC} submission in \gls{ADES} format [MB]}\\
\hline\hline
per night & all obs. & discoveries & known obj. \\\hline
median  & 252 &149  &84  \\
mean & 296 & 192 & 92   \\
std. dev. & 244 & 173& 69 \\
maximum & 1,419 & 1,018 & 326 \\
\hline\hline\\
\end{tabular}
\end{center}
\caption{Nightly data volumes for the submission of \gls{LSST} \gls{MBA} observations, discoveries and currently known object observations to the Minor Planet \gls{Center} in MegaBytes.}
\label{tab:data_stats}
\end{table}

\clearpage
\section{Other Solar System \gls{Object} populations}\label{sec:other}
Similar to the inner Solar System, size frequency distributions for SSOs in the outer solar system are not well understood.
For Trojans and Trans-Neptunian Objects (TNOs) only cumulative completeness estimates are available at this point. Those are presented in Figure \ref{fig:allcum}.
Until updated figures become available we suggest to use the previous predictions of 40k Trojan and 280k \gls{TNO} discoveries from \citet{jones2015asteroid}.
Interstellar objects and comets have not yet been investigated in detail.

\begin{figure}[tb!]
\begin{center}
\includegraphics[width=0.60\linewidth]{figs/allcum.png}
\end{center}
\caption{Cumulative completeness for various solar system populations with the Feature Based Scheduler 1.2 baseline.
The two lines show results with visit pairs in the same filter vs. pairs in different filters (TNOs with C colors)}
\label{fig:allcum}       % Give a unique label
\end{figure}

\clearpage
\section{Conclusions} \label{sec:conclusions}

LSST is going to be the largest contributor of Solar System object observations and discoveries to be processed by the \gls{IAU} Minor Planet \gls{Center}.
The study at hand shows that
\begin{itemize}
\item \gls{LSST} is expected to discover between 49k-93k near-Earth asteroids (NEOs)  with H$<25$mag depending on the actual size-frequency distribution.
\item Observations and discoveries of Main belt asteroids (MBAs) will dominate \gls{LSST} \gls{SSO} data submitted to the \gls{IAU} Minor Planet \gls{Center}.
\item The number of MBAs discovered over 10 years of \gls{LSST} likely ranges between 4.8M and 5.4M and is unlikely to be higher than 220M.
\item The majority of \gls{MBA} discoveries should be made within the first 2 to 3 years of the \gls{LSST} survey.
\item The number of \gls{MBA} discoveries per night can be up to 100 000 with an average of 2000 for the most likely size-frequency distribution (population model A). Even in the unlikely case of a \gls{TNO}-like size-frequency distribution in the Main Belt we expect no more than 3 million discoveries per night on a handful of nights during the survey.
\item The vast majority of asteroids observed with \gls{LSST} have two and more observations per night.
\item The "three nighter" paradigm for detecting SSOs leads to unattributable observations, since not all objects are observed multiple times on three separate nights. However, the overall number of these observations is low. After 6 months only roughly 15\% of asteroid observations cannot be attributed to known objects.
\item Nightly data volumes delivered to the Minor Planet \gls{Center} are generally on the order of hundreds of MBs and only rarely exceed \gls{GB}s.
\end{itemize}
